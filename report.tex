\documentclass[12pt]{article}
%\usepackage[swedish]{babel}
%\usepackage[latin1]{inputenc}
\usepackage{graphicx}
\usepackage{fullpage}
\usepackage{pdfpages}
\usepackage{float}
\usepackage{listings}
\usepackage{color}
\title{Reprt lab 1}
\author{Group 22}
\date{\today}

\setlength{\parindent}{0pt}
\setlength{\parskip}{2ex}

\begin{document}

%\includepdf{Forsattsblad_SigSys.pdf}

\pagebreak

\maketitle

\pagebreak

\tableofcontents

\pagebreak

\section{Introduction}
This is a report for a laboration in the course Digital Signal Processing TSRT78.
The laboration consist of three assignments.
All of them are related to AR models and estimation of them.
The assignments are implemented in Matlab.
Input data for the assignments where collected by recording sounds and importing them into Matlab.
These sounds where then used to complete the assignments.

\section{Whistle}
\subsection{Task}


\subsection{Theoretical overview}

\subsection{Practical Execution}

\subsection{Results}

\section{Vowel}

\subsection{Task}
In this assignment an AR model for a the vowels 'a' and 'o' is to be estimated.
The required order of the model is unknown and to find an appropriate order is part of the assignment.
The estimated models should also be validated and different order models should be compared.

\subsection{Theoretical overview}
An AR model of the sound can be found in the same way as was done in the first assignment.
The problem this time is that a good order for the model must be found.
Different model orders need to be validated and compared.
A good way to do this is splitting the available data into two sets one set is used to estimate the model and the other set is used for validation.

The vowel that are estimated are voiced sounds.
This means that a pulse train is a good input signal to the model.

\subsection{Practical Execution}

\subsection{Results}

\section{Sentence}

\subsection{Task}

\subsection{Theoretical overview}

\subsection{Practical Execution}

\subsection{Results}

\section{Conclusion}



%\begin{figure}[H]
%  \centering
%  \includegraphics[width=14cm]{flow_chart.png}
%  \caption{Flow chart of Slotted ALOHA simulation.}
%\end{figure}



\clearpage
\definecolor{dkgreen}{rgb}{0,0.6,0}
\definecolor{gray}{rgb}{0.5,0.5,0.5}
\definecolor{mauve}{rgb}{0.58,0,0.82}
\lstset{
  title=\lstname,
  basicstyle=\footnotesize\ttfamily,
  keywordstyle=\color{blue},          % keyword style
  commentstyle=\color{dkgreen},       % comment style
  stringstyle=\color{mauve},         % string literal style
  showstringspaces=false,         % underline spaces within strings
}
%\lstinputlisting[language=C++]{slotted_aloha.cc}
\end{document}

