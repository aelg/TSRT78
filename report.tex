\documentclass[12pt]{article}
%\usepackage[swedish]{babel}
%\usepackage[latin1]{inputenc}
\usepackage{graphicx}
\usepackage{fullpage}
\usepackage{pdfpages}
\usepackage{float}
\usepackage{listings}
\usepackage{color}
\title{Reprt lab 1}
\author{Group 22}
\date{\today}

\setlength{\parindent}{0pt}
\setlength{\parskip}{2ex}

\begin{document}

%\includepdf{Forsattsblad_SigSys.pdf}

\pagebreak

\maketitle

\pagebreak

\tableofcontents

\pagebreak

\section{Introduction}
This is a report for a laboration in the course Digital Signal Processing TSRT78.
The laboration consist of three assignments.
All of them are related to AR models and estimation of them.
The assignments are implemented in Matlab.
Input data for the assignments where collected by recording sounds and importing them into Matlab.
These sounds where then used to complete the assignments.

\section{Whistle}
\subsection{Task}
The first assignment consist of two parts. The input data is a recorded whistle which purity compared to a sine wave is estimated in the first part. The second part is about estimating the whistle using a second order AR. The signal is shown in .. %infoga figurnumrering.

\begin{figure}[H]
\centering
\includegraphics[width=14cm]{Fouriertranswhis.png}
\caption{The fouriertransform of the signal processed in the first assignment}
\end{figure}

\subsection{Theoretical overview}

\subsubsection{Energy and harmonic distortion}
A good way to approxiamate the purity of a periodic signal is to calculate the harmonic distorion. It is defined:

 \begin{equation}Dist = 1-\frac{E_{dom. freq.}}{E_{tot.}}\end{equation}

The energy $E$ for a signal is defined as

\begin{equation}E=\int_{0}^T \vert y(t)\vert ^2 dt=\frac{1}{2\pi}\int_{\frac{-\pi}{T}}^{\frac{\pi}{T}}\vert Y(\theta)\vert ^2d\theta \end{equation}
%ssjukt fula integraler. bör fixas

where $Y(\theta)$ is the fouriertransform of $y(t)$ and T is the length od the signal. Since the signals in this laboratory are timediscrete, the energy can not be calculated this way. The integrals can instead be approximated as a sum of $N$ elements that have the width $T/N$, where $T$ is the length of the integrationinterval.

\begin{equation}
E=\int_{0}^T \vert y(t)\vert^2 dt\approx \frac{T}{N}\sum_{n=0}^{N-1}\vert y(n) \vert^2 
\label{A} 
\end{equation}

and

\begin{equation}
E=\frac{1}{2\pi}\int_{\frac{-\pi}{T}}^{\frac{\pi}{T}}\vert Y(\theta)\vert ^2d\theta \approx \frac{2\pi}{NT}\frac{1}{2\pi}\sum_{n=0}^{N-1}\vert Y(\theta) \vert^2=\frac{1}{NT}\sum_{n=0}^{N-1}\vert Y(\theta) \vert^2 
\end{equation}

\subsubsection{AR}
AR means auto-regressive and is a way of modeling signals. $AR(n)$ is defined as

\begin{equation} y(t)+a_1y(t-1)+...a_ny(t-n)=e(t) \end{equation}

where $e(t)$ is white noise.

With the notation

\begin{equation}
(\varphi(t) =(-y(t-1) ... -y(t-n))^T,\theta = (a_1 ... a_n)^T 
\end{equation}

$y(t)$ can be defined as the linear model

\begin{equation}
y(t)=\varphi^T(t)\theta+ e(t)
\end{equation}

The optimal use of $\theta$ is

\begin{equation}
\hat{\theta}=arg_{\theta}min \frac{1}{N}\sum_{t=1}^{\infty}(y(t)-\varphi^T(t)\theta)^2
\end{equation}

which is given by

\begin{equation}
\hat{\theta}=\left(\sum_{t=1}^N\varphi(t)\varphi^T(t)\right)^{-1}\left(\sum_{t=1}^N\varphi(t)y(t)\right)
\end{equation}


\subsection{Practical Execution}
The distortion of the signal can be calculated in two different ways. Either the energy of the signal is calculated in the time-domain or in the frequency domain. In this lab both methods are handeled.

The energy for The whole signal in the the time-domain is calculated according to equation \ref{A}. The energy for the dominating frequencies is acquired by filtering the signal with a bandpassfilter with the dominating frequencies in the passband. The dominating frequencies can be approximated to 1080-1160 Hz. 

\begin{equation}
E_{tot.} \approx 0.0014, E_{dom. freq.\approx 0.0013, Dist_{time} = 0.0511

\end{equation}

\begin{equation}
E_{total}=



\subsection{Results}

\section{Vowel}

\subsection{Task}

\subsection{Theoretical overview}

\subsection{Practical Execution}

\subsection{Results}

\section{Sentence}

\subsection{Task}

\subsection{Theoretical overview}

\subsection{Practical Execution}

\subsection{Results}

\section{Conclusion}



%\begin{figure}[H]
%  \centering
%  \includegraphics[width=14cm]{flow_chart.png}
%  \caption{Flow chart of Slotted ALOHA simulation.}
%\end{figure}



\clearpage
\definecolor{dkgreen}{rgb}{0,0.6,0}
\definecolor{gray}{rgb}{0.5,0.5,0.5}
\definecolor{mauve}{rgb}{0.58,0,0.82}
\lstset{
  title=\lstname,
  basicstyle=\footnotesize\ttfamily,
  keywordstyle=\color{blue},          % keyword style
  commentstyle=\color{dkgreen},       % comment style
  stringstyle=\color{mauve},         % string literal style
  showstringspaces=false,         % underline spaces within strings
}
%\lstinputlisting[language=C++]{slotted_aloha.cc}
\end{document}

